\documentclass[a4paper]{article}

%% Language and font encodings
\usepackage[english]{babel}
\usepackage[utf8x]{inputenc}
\usepackage[T1]{fontenc}

%% Sets page size and margins
\usepackage[a4paper,top=3cm,bottom=2cm,left=3cm,right=3cm,marginparwidth=1.75cm]{geometry}

%% Useful packages
\usepackage{amsmath}
\usepackage{graphicx}
\usepackage[colorinlistoftodos]{todonotes}
\usepackage[colorlinks=true, allcolors=blue]{hyperref}

\title{The Programmer as Navigator}
\author{Charles W.Bachman\\\\Summarised by  Nikhil Reddy. P, CS16B018}

\begin{document}
\maketitle

\section{Introduction}
\hspace{1cm}A new basis for understanding is
available in the area of information systems. It is
achieved by a shift from a computer centered to the
database centered point of view. This new understanding will lead to new solutions to our database problems and speed our conquest of the n-dimensional data structures which best model the complexities of the real world. 

\section{Into the Database}
\begin{itemize}
\item The earliest databases, initially implemented on
punched cards with sequential file technology.
\item Sequential file technology,Start with the value of the primary
data key, of the record of interest, and pass each record
in the file through core memory until the desired record,
or one with a higher key, is found.
\item The revolution in
thinking is changing the programmer from a stationary
viewer of objects passing before him in core into a
mobile navigator who is able to probe and traverse a
database at will.
\item Direct access storage devices also opened up new
ways of record retrieval by primary data key. The first
was called randomizing, calculated addressing, or hashing.
\item The programmer who has advanced from sequential
file processing to either index sequential or randomized
access processing has greatly reduced his access time.
\item The programmer's
training should be a full-fledged navigator in an n-dimensional dataspace.
\end{itemize}
\section{Database Management systems}
Database Management involves all aspects of storing, retrieving, modifying, and deleting data in the files on personnel and production, airline reservations,-data which is used repeatedly and updated as new information becomes available. These files are mapped through some storage structure onto magnetic tapes or disk packs and the drives that support them.
\begin{itemize}
\item Database management has two main functions.
\end{itemize}
\begin{enumerate}
\item First is the inquiry or retrieval activity that reaccesses previously stored data in order to determine the recorded status of some real world entity or relationship. This data has previously been stored by some other job,
seconds, minutes, hours, or even days earlier, and has
been held in trust by the database management system.It has a continuing responsibility to maintain data between the time when it was stored and the time it is subsequently required for retrieval.
\item The second activity is to update, which includes the original storage of data, its repeated modification as things change, and ultimately,its deletion from the system when the data is no longer needed.
\end{enumerate}
\begin{itemize}
\item In addition to a record's primary key, it is frequently
desirable to be able to retrieve records on the basis of
the value of some other fields. For example, it may be
desirable, in planning ten-year awards, to select all the
employee records with the "year-of-hire" field value
equal to 1964. Such access is retrieval by secondary data key.
\item This equality of primary and secondary data key
fields reflects real world relationships and provides a
way to reestablish these relationships for computer
processing purposes. The use of the same data value as
a primary key for one record and as a secondary key
for a set of records is the basic concept upon which
data structure sets are declared and maintained.
\item The Integrated Data Store (I-D-S) systems and all other
systems based on its concepts consider their basic con-
tribution to the programmer to be the capability to
associate records into data structure sets and the cap-
ability to use these sets as retrieval paths.
\item Performance is enhanced by the so-called "clustering" ability of databases where the owner and some or most of the members records of a set are physically stored and accessed together on the same block or page. These systems have
been running in virtual memory since 1962.
\item The significant functional and performance
advantage is to be able to specify the order of retrieval
of the records within a set based upon a declared sort
field or the time of insertion.
\item As navigator he must brave dimly percieved shoals and reefs in his sea,which are created because he has to navigate in a shared database environment.
\end{itemize}
\section{Shared Access}
\begin{itemize}
\item Shared access is a new and complex variation of
multiprogramming or time sharing, which were invented to permit shared, but independent, use of then computer resources.
\item Shared access is a specialized version of multiprogram-
ming where the critical, shared resources are the records
of the database
\item The pressures to use shared access are tremendous.Two problems would maintain the pressure for successful shared access. The first is the trend toward the integration of many single purpose files into a few integrated databases; the second is the trend toward interactive processing where the processor can only advance a job as fast as the manually created input messages allow.
\item Of the two main functions of database management,
inquiry and update, only update creates a potential
problem in shared access.Once a single job begins to up-
date the database, a potential for trouble exists.
\item The two basic causes of trouble in shared access are
interference and contamination.
\begin{enumerate}
\item Interference is defined as the negative effect of the updating activity of one job upon the results of another.One job running an accounting trial balance while another was posting transactions illustrates the interference priblem
\item Contamination is defined as
the negative effect upon a job which results from a com-
bination of two events: when another job has aborted
and when its output has already been read by the first job.
\end{enumerate}
\item A critical question in designing solutions to the
shared access problem is the extent of visibility that the
application programmer should have.
\end{itemize}

\section{Interesting Projects}
\begin{enumerate}
\item  Z511 Project 
\begin{itemize}
\item Description  

The project involves designing a complete database management system to address a practical database need and implementing a relational database based on that design.  Your database system should be designed to perform general information management tasks such as systematic collection, update, and retrieval of information for a small organization (e.g. music/video/book store).  Students should work in a group of 3 or less.
\end{itemize}
\item Khipu Database Project
\begin{itemize}
\item Description


The Khipu Database Project began in the fall of 2002, with the goal of collecting all known information about khipu into one centralized repository. Having the data in digital form allows researchers to ask questions about khipu which up until now would have been very difficult, if not impossible, to answer.
\end{itemize}
\end{enumerate}


\section{Future Scope}
\begin{enumerate}
\item A Vision for a Converged Database
\begin{itemize}
\item An ideal database architecture would support multiple data models, languages, processing paradigms and storage formats within the one system.   Application requirements that dictate a specific database feature should be resolved as configuration options or pluggable features within a single database management system, not as choices between disparate database architectures.
\end{itemize}
\item Disruptive Database Technologies
\begin{itemize}
\item Extrapolating existing technologies is a useful pastime, and is often the only predictive technique available.  However, history teaches us that technologies don’t always continue upon an existing trajectory.  Disruptive technologies emerge which create discontinuities that cannot be extrapolated and cannot always be fully anticipated.
\item There are a few computing technology trends which extend beyond database architecture and which may impinge heavily on the databases of the future.They are : 
\begin{enumerate}
\item Universal Memory : A technology should arise that simultaneously provides acceptable economics for mass storage and latency then we might see an almost immediate shift in database architectures.  Such a universal memory would provide access speeds equivalent to RAM together with the durability, persistence and storage economics of disk.
\item Quantum Computing : Quantum computers could break existing private/public key encryption schemes seems increasingly likely, while quantum key transmission already provides a tamper-proof mechanism for transmitting certificates over distances within a few hundreds of kilometers. 
\end{enumerate}
\end{itemize}
\end{enumerate}
\section{Reference}
\hspace{1cm} \href{http://amturing.acm.org/award_winners/bachman_9385610.cfm}{Page acm turing award to Charles W.Bachman,1973}

\end{document}